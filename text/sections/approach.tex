\section{Approach}
\label{sec:approach}

First, we introduce the small-portfolio problem.
Next, we present multiple approaches to solve it.

\subsection{Small-Portfolio Problem}

\paragraph{Given Data}

\begin{align*}
	S &= \{s_1, \dots, s_n\} \tag*{Solvers}\\
	I &= \{i_1, \dots, i_m\} \tag*{Instances}\\
	r &: S \times I \rightarrow \mathbb{N} \cup \{\bot\} \tag*{Runtimes (censored)}\\
	T &\in \mathbb{N} \tag*{Timeout}\\
	r_T(s,i) &:= \begin{cases}
		2*T & \text{if }r(s,i) = \bot\\
		r(s,i) & \text{otherwise}
	\end{cases} \tag*{Penalized Runtimes}
\end{align*}

Let $S$ be a set of solvers and $I$ be a set of SAT instances.
Let $r(s,i)$ be the runtimes of the solvers on the instances.
If a solver encounters an error, e.g., runs out of memory, or takes longer than time $T$ on an instance, the runtime is set to $\bot$ first.
Next, we penalize these missing runtimes with the double timeout, i.e., we use a PAR2 score.
We use these penalized runtimes $r_T(s,i)$ to score the solvers.

\paragraph{Target Function}

\begin{align*}
	c_{T} &: 2^S \rightarrow \mathbb{N}\\
	c_{T}(P) &:= \begin{cases}
		|I|*2*T & \text{if }P=\emptyset\\
		\sum_{i \in I}{\min\{r_T(s,i) \mid s \in P\}} & \text{otherwise}
	\end{cases} \tag*{Portfolio Cost}
\end{align*}

The cost of a solver equals its penalized runtime, summed over all instances.
A portfolio $P \subseteq S$ is a set of solvers.
To compute the cost of a portfolio $c_{T}(P)$, we assume to have an oracle that always chooses the fastest solver for each instance.
This is the \emph{virtual-best solver} (VBS).

In reality, one may train a prediction model that recommends a solver for each instance.
Such a prediction model uses features of the SAT instances to make its recommendations.
Only if the prediction model always recommends the fastest solver out of $P$ for each instance, the portfolio has the same costs as the VBS for $P$.
Else, the costs are higher, i.e., $c_{T}(P)$ is a lower bound for the actual portfolio costs.
In the worst case, the prediction model always recommends the slowest solver out of $P$ for each instance.
This serves as an upper bound for actual portfolio costs and it the \emph{virtual worst solver} (VWS).

\paragraph{Optimization Problem}

\begin{align*}
	\min_P \quad & c_{T}(P)\\
	s.t. \quad & |P| \leq k
\end{align*}

The small portfolio-problem is to find a portfolio with minimum costs that contains at most $k$ solvers.
For each solver out of $S$, one needs to decide whether it becomes part of the portfolio $P$ or not.
