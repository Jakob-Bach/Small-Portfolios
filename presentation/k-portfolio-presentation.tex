% Customized "beamer" class
% Template cloned from https://git.scc.kit.edu/i43/dokumentvorlagen/praesentationen/beamer | commit: 5c6fe51d431425942a350e8a439bb4cef08f0275 (2022-06-28)
% Due to unclear licensing, we do not provide the files "sdqbeamer.cls" and "logos/kitlogo_en_rgb.pdf" (need to be added manually)
\documentclass[en]{sdqbeamer}
% - remove animation roll-out: handout (general "beamer" option, not specific for this class)
% - layout options: 16:9 (default), 16:10, 4:3
% - footer font size options: bigfoot (default), smallfoot (KIT layout)
% - navigation bar options: navbarinline (default), navbarinfooter, navbarside, navbaroff, navbarkit (off + smallfoot)
% - language: de (default), en

\titleimage{}

\grouplogo{}

\groupname{}
%\groupnamewidth{50mm} % default

\title[A Comprehensive Study of k-Portfolios of Recent SAT Solvers]{A Comprehensive Study of k-Portfolios of Recent SAT Solvers} % [footer]{title slide}
\subtitle{SAT 2022 | Haifa, Israel}
\author[\underline{Jakob Bach}, Markus Iser, and Klemens Böhm]{\underline{Jakob Bach}, Markus Iser, and Klemens Böhm} % [footer]{title slide}
\date[2022-08-02]{August 2, 2022} % [footer]{title slide}

%\usepackage{amsmath} % mathematical symbols and equations; apparently pre-loaded
%\usepackage{amssymb} % mathematical symbols; apparently pre-loaded
\usepackage[style=numeric, backend=bibtex]{biblatex}  % original template uses "biber" as backend
%\usepackage{graphicx} % plots; apparently pre-loaded
%\usepackage{hyperref} % links and URLs; apparently pre-loaded
\usepackage{subcaption} % figures with multiple sub-figures and sub-captions

\addbibresource{k-portfolio-presentation.bib}

\setlength{\leftmargini}{0.2cm} % change default identation (so items are left-aligned to boxes)
\setlength{\leftmarginii}{0.3cm} % 2nd level identation
\setlength{\leftmarginiii}{0.3cm} % 3rd level identation

\setbeamercovered{invisible} % use "transparent" to show later content of animated slide in gray
\setbeamertemplate{enumerate items}[default] % do not use the ugly colored circles

\begin{document}

\KITtitleframe

\section{Experiments}

\begin{frame}[t]{Experimental Design}
	\begin{itemize}
		\item Two datasets (from Main Tracks of recent SAT Competitions):
		\begin{enumerate}[1)]
			\item \emph{SC2020} (316 instances, 48 solvers)~\cite{balyo2020proceedings}
			\item \emph{SC2021} (325 instances, 46 solvers)~\cite{balyo2021proceedings}
			%JB: took all solvers, but removed instances not solved any of them
		\end{enumerate}
		\begin{itemize}
			\item 138 features from feature extractor of SATzilla~2012~\cite{xu2012features, xu2012satzilla2012}
			%JB: features from twelve categories, simple ones (like number of instances) to complicated ones (like variable-clause graph node degree)
			%JB: missing values due to timeouts and memouts replaced with out-of-range value
			\item Instance features and solver runtimes retrieved from GBD~\cite{iser2020collaborative}
		\end{itemize}
		\pause
		\vspace{\baselineskip}
		\item Four solution approaches:
		%JB: all of them run for both datasets and all k (from 1 to number of solvers)
		\begin{itemize}
			\item \emph{Optimal solution} via MIP solving~\cite{python-mip}
			%JB: that's the exact approach, others are heuristics
			%JB: package "mip" uses the solver "COIN-OR branch-and-cut" (Cbc) internally
			\item \emph{Beam search} with beam width $w \in \{1, 2, 3, \dots, 10, 20, 30, \dots, 100\}$
			\item \emph{K-best}
			\item \emph{Random sampling} with 1000 repetitions
		\end{itemize}
		\pause
		\vspace{\baselineskip}
		\item Two multi-class prediction models: Random forests~\cite{breiman2001random, scikit-learn} and XGBoost~\cite{xgboost} with 100 trees each
		%JB: ensemble tree models: powerful and can learn non-linear dependencies (RFs also used in SATzilla 2012)
		%JB: in preliminary experiments, also tried other models (e.g., kNN, untuned neural network) -> worse performance
		%JB: in preliminary experiments, also regression, instance-weighted classififcation, one-vs-one classification -> worse performance
	\end{itemize}
\end{frame}

\appendix
\beginbackup % subsequent slides do not impact overall slide count

\begin{frame}[t, allowframebreaks]{References}
	\printbibliography
\end{frame}

\backupend

\end{document}
